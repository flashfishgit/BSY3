\documentclass{report}
\usepackage{pdfpages}
\usepackage[a4paper, top=2cm, bottom=2cm, left=2cm, right=2cm]{geometry}
\usepackage{fancyhdr}
\usepackage{amsmath} % Für mathematische Symbole
\usepackage{amssymb} % Für \mathbb
\usepackage[linesnumbered,ruled]{algorithm2e} % Für Algorithmus-Umgebung
\usepackage{ulem}
\usepackage{xcolor}
\usepackage{array}
\usepackage{graphicx}
\usepackage{listings}
\usepackage{xcolor}
\usepackage{naive-ebnf}
\usepackage[utf8]{inputenc}
\usepackage[absolute,overlay]{textpos}
\usepackage{tcolorbox}

\lstdefinestyle{cppstyle}{
    language=C++,             % Specify the language
    basicstyle=\ttfamily,     % Set the font to typewriter
    basicstyle=\ttfamily\footnotesize,
    keywordstyle=\color{blue}\bfseries, % Keywords in bold blue
    commentstyle=\color{green},         % Comments in green
    stringstyle=\color{red},            % Strings in red
    numbers=left,            % Line numbers on the left
    numberstyle=\tiny\color{gray}, % Line numbers in tiny gray font
    stepnumber=1,             % Line number step
    breaklines=true,          % Automatic line breaking
    tabsize=4,                % Set tab size
    showstringspaces=false,   % Don't show spaces in strings
    frame=single,             % Add a frame around the code
}

\newcommand{\name}{Marco Söllinger}
\newcommand{\fach}{SEN2}
\newcommand{\topic}{Übung 1}
\newcommand{\uebungangabe}{Uebung1.pdf}

\newcommand{\matnr}{s2410306011}
\newcommand{\uebungsgruppe}{Gruppe 1}
\newcommand{\aufwand}{2}


\pagestyle{fancy}
\normalem 
\fancyhead[R]{Marco Söllinger}  

\begin{document}
\includepdf[pages=1,pagecommand={
			\begin{textblock*}{5cm}(5.5cm, 6.9cm)
				\textbf{\name}
			\end{textblock*}
			\begin{textblock*}{5cm}(5.5cm, 8.4cm)
				\textbf{\matnr}
			\end{textblock*}
			\begin{textblock*}{5cm}(5.5cm, 9.8cm)
				\textbf{\uebungsgruppe}
			\end{textblock*}
			\begin{textblock*}{5cm}(15.3cm, 6.9cm)
				\textbf{\aufwand}
			\end{textblock*}
		}]{Uebung5.pdf}
\includepdf[pages=2-]{Uebung5.pdf}

\section*{Beispiel 1}
Es wird keine Lösungsidee gefordert, deshalb wurde der Code entsprechend kommentiert. \\\\
Es ist nur darauf zu achten das speziel Fälle wie escaptes " beachtet werden müssen.\\
Wenn das Program nicht eine ganze Zeile auf einmal auslesen kann, dann bricht es ab.\\
Das macht die line Analyse deutlich einfacher und es sollten noramlerweise nie in einem Code 512 chars in einer Zeile sein.

\subsection*{1.1 Code}
\lstinputlisting[style=cppstyle, title=\texttt{ccheck.c} ]{../ccheck.c}


\subsection*{1.2 Test}
Zum Testen wurde ein bash script erstellt, welches die verschiedenen Faelle testet.\\\\
Die einzelen Terminal outputs wurden in ein File geschrieben.\\

\lstinputlisting[style=cppstyle, title=\texttt{test.sh} ]{../test.sh}

\begin{lstlisting}[style=cppstyle, title=\texttt{Terminal Output}]
flashfish@fedora ~/D/R/F/B/Uebung05 (main)> ./test.sh
Run with no arguments:
error in commandline -> ./ccheck fileName
########################################
Run with two argument:
error in commandline -> ./ccheck fileName
########################################
Run with no existing file:
error in fopen: No such file or directory
########################################
Run on test_00_error.txt:
########################################
Run on test_01_error.txt
########################################
Run on test_02_ok.txt
########################################
Test buffer overflow
error: line buffer overflow, increase BUFFER_SIZE: Success
########################################
Run on ccheck.c
\end{lstlisting}

\lstinputlisting[style=cppstyle, title=\texttt{test\_00\_error.txt} ]{../test_00_error.txt}

\lstinputlisting[style=cppstyle, title=\texttt{test\_00\_output.txt} ]{../test_00_output.txt}

\lstinputlisting[style=cppstyle, title=\texttt{test\_01\_error.txt} ]{../test_01_error.txt}

\lstinputlisting[style=cppstyle, title=\texttt{test\_01\_output.txt} ]{../test_01_output.txt}

\lstinputlisting[style=cppstyle, title=\texttt{test\_02\_ok.txt} ]{../test_02_ok.txt}

\lstinputlisting[style=cppstyle, title=\texttt{test\_02\_output.txt} ]{../test_02_output.txt}

\lstinputlisting[style=cppstyle, title=\texttt{test\_04\_output.txt} ]{../test_04_output.txt}






\end{document}
