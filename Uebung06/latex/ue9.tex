\documentclass{report}
\usepackage{pdfpages}
\usepackage[a4paper, top=2cm, bottom=2cm, left=2cm, right=2cm]{geometry}
\usepackage{fancyhdr}
\usepackage{amsmath} % Für mathematische Symbole
\usepackage{amssymb} % Für \mathbb
\usepackage[linesnumbered,ruled]{algorithm2e} % Für Algorithmus-Umgebung
\usepackage{ulem}
\usepackage{xcolor}
\usepackage{array}
\usepackage{graphicx}
\usepackage{listings}
\usepackage{xcolor}
\usepackage{naive-ebnf}
\usepackage[utf8]{inputenc}
\usepackage[absolute,overlay]{textpos}
\usepackage{tcolorbox}

\lstdefinestyle{cppstyle}{
    language=C++,             % Specify the language
    basicstyle=\ttfamily,     % Set the font to typewriter
    basicstyle=\ttfamily\footnotesize,
    keywordstyle=\color{blue}\bfseries, % Keywords in bold blue
    commentstyle=\color{green},         % Comments in green
    stringstyle=\color{red},            % Strings in red
    numbers=left,            % Line numbers on the left
    numberstyle=\tiny\color{gray}, % Line numbers in tiny gray font
    stepnumber=1,             % Line number step
    breaklines=true,          % Automatic line breaking
    tabsize=4,                % Set tab size
    showstringspaces=false,   % Don't show spaces in strings
    frame=single,             % Add a frame around the code
}

\newcommand{\name}{Marco Söllinger}
\newcommand{\fach}{SEN2}
\newcommand{\topic}{Übung 1}
\newcommand{\uebungangabe}{Uebung1.pdf}

\newcommand{\matnr}{s2410306011}
\newcommand{\uebungsgruppe}{Gruppe 1}
\newcommand{\aufwand}{3}


\pagestyle{fancy}
\normalem 
\fancyhead[R]{Marco Söllinger}  

\begin{document}
\includepdf[pages=1,pagecommand={
			\begin{textblock*}{5cm}(5.5cm, 6.9cm)
				\textbf{\name}
			\end{textblock*}
			\begin{textblock*}{5cm}(5.5cm, 8.4cm)
				\textbf{\matnr}
			\end{textblock*}
			\begin{textblock*}{5cm}(5.5cm, 9.8cm)
				\textbf{\uebungsgruppe}
			\end{textblock*}
			\begin{textblock*}{5cm}(15.3cm, 6.9cm)
				\textbf{\aufwand}
			\end{textblock*}
		}]{Uebung6.pdf}
\includepdf[pages=2-]{Uebung6.pdf}

\section*{Beispiel 1}
Es wird keine Lösungsidee gefordert, deshalb wurde der Code entsprechend kommentiert. \\\\
Das Programm wurde mit einer Makefile kompiliert.\\
Da in der Angabe nicht genau spezifiziert wurde, wie das Programm sich verhalten soll, wenn mehr Parameter als benötigt übergeben werden, wurde entschieden, dass die zusätzlichen Parameter ignoriert werden. \\
Weiters Falls ein Buffer Overflow auftritt, wird eine entsprechende Fehlermeldung ausgegeben und der Buffer wird geleert. \\
Es wird dann noch versucht die restlichen Symbole einzulesen und diese als Befehl zu interpretieren. \\\\
Ein paar Testfaelle wurden in der main.c Datei implementiert, der Rest wurdene manuell im Terminal getestet. \\\\

\subsection*{1.1 Code}
\lstinputlisting[style=cppstyle, title=\texttt{main.c} ]{../main.c}
\lstinputlisting[style=cppstyle, title=\texttt{CommandInterpreter.h} ]{../CommandInterpreter.h}
\lstinputlisting[style=cppstyle, title=\texttt{CommandInterpreter.c} ]{../CommandInterpreter.c}
\lstinputlisting[style=cppstyle, title=\texttt{CommandTable.h} ]{../CommandTable.h}
\lstinputlisting[style=cppstyle, title=\texttt{CommandTable.c} ]{../CommandTable.c}



\subsection*{1.2 Test}
Es wurden folgende Testfälle implementiert und ausgeführt:


\begin{lstlisting}[style=cppstyle, title=\texttt{Terminal Output}]
flashfish@fedora ~/D/R/F/B/Uebung06 (main)> make run
./CommandInterpreter
Must initialize module before use!
Must initialize module before use!
Error in Init(...) -> null pointer given for command table
Must initialize module before use!
Must initialize module before use!
Error in Init(...) -> null pointer given for put-char-function
Must initialize module before use!
Must initialize module before use!
Welcome Command-Interpreter
>help
help 
(press <ESC> to exit)
available commands:
help: show help text
?: show help text
string: [value] first parameter as string
int: [value] first parameter as integer
float: [value] first parameter as float
addint: [value1] + [value1] adds two integer values
OK
>?
? 
(press <ESC> to exit)
available commands:
help: show help text
?: show help text
string: [value] first parameter as string
int: [value] first parameter as integer
float: [value] first parameter as float
addint: [value1] + [value1] adds two integer values
OK
>int 123
int 123 
Parameter: 123
OK
>float 3.14
float 3.14 
Parameter: 3.14
OK
>addint 3 4
addint 3 4 
Result: 7
OK
>int 123 123
int 123 123 
Parameter: 123
OK
>float 123 123
float 123 123 
Parameter: 123
OK
>addint 5
addint 5 
ERROR
>addint
addint 
ERROR
>int
int 
ERROR
>float
float 
ERROR
>help 475
help 475 
(press <ESC> to exit)
available commands:
help: show help text
?: show help text
string: [value] first parameter as string
int: [value] first parameter as integer
float: [value] first parameter as float
addint: [value1] + [value1] adds two integer values
OK
>adf
adf 
Unknown command!

>
\end{lstlisting}


\begin{lstlisting}[style=cppstyle, title=\texttt{Terminal Output}]
flashfish@fedora ~/D/R/F/B/Uebung06 (main)> make run
./CommandInterpreter
Must initialize module before use!
Must initialize module before use!
Error in Init(...) -> null pointer given for command table
Must initialize module before use!
Must initialize module before use!
Error in Init(...) -> null pointer given for put-char-function
Must initialize module before use!
Must initialize module before use!
Welcome Command-Interpreter
>BufferOverflowTest1234567890ABCDEFGHIJKLMNOPQRSTUVWXYZabcdefghijklmnopqrstuvwxyz1234567890ABCDEFGHIJKLMNOPQRSTUVWXYZabcdefghijklmnopqrstuvwxyz1234567890ABCDEFGHIJKLMNOPQRSTUVWXYZabcdefghijklmnopqrstuvwxyz1234567890ABCDEFGHIJKLMNOPQRSTUVWXYZabcdefghijklmnopqrstuvwxyz1234567890ABCDEFGHIJKLMNOPQRSTUVWXYZabcdefghijklmnopqrstuvwxyz1234567890ABCDEFGHIJKLMNOPQRSTUVWXYZabcdefghijklmnopqrstuvwxyz1234567890
Input buffer overflow! Max input length exceeded.
Input buffer overflow! Max input length exceeded.
Input buffer overflow! Max input length exceeded.
xyz1234567890 
Unknown command!

>
\end{lstlisting}


\begin{lstlisting}[style=cppstyle, title=\texttt{Terminal Output}]
flashfish@fedora ~/D/R/F/B/Uebung06 (main) [SIGINT]> make run
./CommandInterpreter
Must initialize module before use!
Must initialize module before use!
Error in Init(...) -> null pointer given for command table
Must initialize module before use!
Must initialize module before use!
Error in Init(...) -> null pointer given for put-char-function
Must initialize module before use!
Must initialize module before use!
Welcome Command-Interpreter
>int 1Param 2Param 3Param 4Param 5Param 6Param 7Param 8Param 9Param 10Param 11Param 12Param 13Param 14Param 15Param 16Param 17Param 18Param 19Param 20Param
Input buffer overflow! Max input length exceeded.
m 18Param 19Param 20Param 
Unknown command!

>
\end{lstlisting}

\begin{lstlisting}[style=cppstyle, title=\texttt{Terminal Output}]
flashfish@fedora ~/D/R/F/B/Uebung06 (main)> make run
clang  -Wall -Wextra -c CommandTable.c -o build/CommandTable.o
clang -g -o CommandInterpreter build/CommandInterpreter.o build/CommandTable.o build/main.o -lm
./CommandInterpreter
Must initialize module before use!
Must initialize module before use!
Error in Init(...) -> null pointer given for command table
Must initialize module before use!
Must initialize module before use!
Error in Init(...) -> null pointer given for put-char-function
Must initialize module before use!
Must initialize module before use!
Welcome Command-Interpreter
>    int 325
int 325
Parameter: 325
OK
>int 		345
int 345
Parameter: 345
OK
>int Ha345
int Ha345
ERROR
>int 352Hal
int 352Hal
ERROR
>float 3h3
float 3h3
ERROR
>float 3.0
float 3.0
Parameter: 3.0
OK
>string Hallo353
string Hallo353
Parameter: Hallo353
OK
>string 343
string 343
Parameter: 343
OK
>addint 34h 3
addint 34h 3
ERROR
>add int 3 3h
add int 3 3h
Unknown command!

>addint 3 3h
addint 3 3h
ERROR
>
\end{lstlisting}

\end{document}
