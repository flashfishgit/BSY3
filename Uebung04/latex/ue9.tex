\documentclass{report}
\usepackage{pdfpages}
\usepackage[a4paper, top=2cm, bottom=2cm, left=2cm, right=2cm]{geometry}
\usepackage{fancyhdr}
\usepackage{amsmath} % Für mathematische Symbole
\usepackage{amssymb} % Für \mathbb
\usepackage[linesnumbered,ruled]{algorithm2e} % Für Algorithmus-Umgebung
\usepackage{ulem}
\usepackage{xcolor}
\usepackage{array}
\usepackage{graphicx}
\usepackage{listings}
\usepackage{xcolor}
\usepackage{naive-ebnf}
\usepackage[utf8]{inputenc}
\usepackage[absolute,overlay]{textpos}
\usepackage{tcolorbox}

\lstdefinestyle{cppstyle}{
    language=C++,             % Specify the language
    basicstyle=\ttfamily,     % Set the font to typewriter
    basicstyle=\ttfamily\footnotesize,
    keywordstyle=\color{blue}\bfseries, % Keywords in bold blue
    commentstyle=\color{green},         % Comments in green
    stringstyle=\color{red},            % Strings in red
    numbers=left,            % Line numbers on the left
    numberstyle=\tiny\color{gray}, % Line numbers in tiny gray font
    stepnumber=1,             % Line number step
    breaklines=true,          % Automatic line breaking
    tabsize=4,                % Set tab size
    showstringspaces=false,   % Don't show spaces in strings
    frame=single,             % Add a frame around the code
}

\newcommand{\name}{Marco Söllinger}
\newcommand{\fach}{SEN2}
\newcommand{\topic}{Übung 1}
\newcommand{\uebungangabe}{Uebung1.pdf}

\newcommand{\matnr}{s2410306011}
\newcommand{\uebungsgruppe}{Gruppe 1}
\newcommand{\aufwand}{4}


\pagestyle{fancy}
\normalem 
\fancyhead[R]{Marco Söllinger}  

\begin{document}
\includepdf[pages=1,pagecommand={
			\begin{textblock*}{5cm}(5.5cm, 6.9cm)
				\textbf{\name}
			\end{textblock*}
			\begin{textblock*}{5cm}(5.5cm, 8.4cm)
				\textbf{\matnr}
			\end{textblock*}
			\begin{textblock*}{5cm}(5.5cm, 9.8cm)
				\textbf{\uebungsgruppe}
			\end{textblock*}
			\begin{textblock*}{5cm}(15.3cm, 6.9cm)
				\textbf{\aufwand}
			\end{textblock*}
		}]{Uebung4.pdf}
%\includepdf[pages=2-]{Uebung2.pdf}

\section*{Beispiel 1}
In dieser Aufgabe soll ein Programm erstellt werden, welches den Inhalt einer Textdatei in eine andere Textdatei kopiert werden.\\
Dabei muessen die speziall Faelle beachtet werden:\\
- Die Quelldatei existiert nicht.\\
- Die Zieldatei existiert nicht(erstellen).\\
- Argumente Anzahl stimmt nicht.\\
- Quelldatei und Zieldatei sind identisch.\\\\
Die Implementierung darf nur direkte Systemaufrufe verwenden (open, read, write, close, ...).\\
Mit der Ausnahme von der string.h Bibliothek fuer String Operationen.\\\\

\subsection*{1.1 Code}
\lstinputlisting[style=cppstyle, title=\texttt{append.c} ]{../append.c}


\subsection*{1.2 Test}
Zum Testen wurde ein bash script erstellt, welches die verschiedenen Faelle testet.\\\\

\lstinputlisting[style=cppstyle, title=\texttt{test.sh} ]{../test.sh}

\begin{lstlisting}[style=cppstyle, title=\texttt{Terminal Output}]
flashfish@fedora-4 ~/D/R/F/B/Uebung04 (main)> ./test.sh
Run with no arguments:
append.c:71: useage: ./append.c <inputfile> <outputfile>
########################################
Run with one argument:
append.c:71: useage: ./append.c <inputfile> <outputfile>
########################################
Run with three argument:
append.c:71: useage: ./append.c <inputfile> <outputfile>
########################################
Run with non existent input file:
append.c:89: error in open: nonExist.txt
########################################
Run with a file without content (Input: test1.txt | Output: empty.txt):
Total bytes written: 0
########################################
Run with indentical Input and Output file (Input: test1.txt | Output: test1.txt):
append.c:76: error: input and output file must be different
########################################
Run with existent output file (Input: test2.txt | Output: output2.txt):
Total bytes written: 21
########################################
Run with existent output file and appent (Input: test3.txt | Output: output3.txt):
Total bytes written: 34
########################################
Run with nonexistent output file (Input: test4.txt | Output: output4.txt):
Total bytes written: 17
########################################
Run with large output file (Input: test5.txt | Output: output5.txt):
Total bytes written: 11000
\end{lstlisting}

\lstinputlisting[style=cppstyle, title=\texttt{test1.txt} ]{../test1.txt}

\lstinputlisting[style=cppstyle, title=\texttt{empty.txt} ]{../empty.txt}

\lstinputlisting[style=cppstyle, title=\texttt{test2.txt} ]{../test2.txt}

\lstinputlisting[style=cppstyle, title=\texttt{output2.txt} ]{../output2.txt}

\lstinputlisting[style=cppstyle, title=\texttt{test3.txt} ]{../test3.txt}

\lstinputlisting[style=cppstyle, title=\texttt{output3.txt} ]{../output3.txt}

\lstinputlisting[style=cppstyle, title=\texttt{test4.txt} ]{../test4.txt}

\lstinputlisting[style=cppstyle, title=\texttt{output4.txt} ]{../output4.txt}

Test5.txt und Output5.txt wurde nicht angehaengt, da es sich um eine grosse Datei handelt (11.000 Zeichen).\\





\end{document}
