\documentclass{report}
\usepackage{pdfpages}
\usepackage[a4paper, top=2cm, bottom=2cm, left=2cm, right=2cm]{geometry}
\usepackage{fancyhdr}
\usepackage{amsmath} % Für mathematische Symbole
\usepackage{amssymb} % Für \mathbb
\usepackage[linesnumbered,ruled]{algorithm2e} % Für Algorithmus-Umgebung
\usepackage{ulem}
\usepackage{xcolor}
\usepackage{array}
\usepackage{graphicx}
\usepackage{listings}
\usepackage{xcolor}
\usepackage{naive-ebnf}
\usepackage[utf8]{inputenc}
\usepackage[absolute,overlay]{textpos}
\usepackage{tcolorbox}

\lstdefinestyle{cppstyle}{
    language=C++,             % Specify the language
    basicstyle=\ttfamily,     % Set the font to typewriter
    basicstyle=\ttfamily\footnotesize,
    keywordstyle=\color{blue}\bfseries, % Keywords in bold blue
    commentstyle=\color{green},         % Comments in green
    stringstyle=\color{red},            % Strings in red
    numbers=left,            % Line numbers on the left
    numberstyle=\tiny\color{gray}, % Line numbers in tiny gray font
    stepnumber=1,             % Line number step
    breaklines=true,          % Automatic line breaking
    tabsize=4,                % Set tab size
    showstringspaces=false,   % Don't show spaces in strings
    frame=single,             % Add a frame around the code
}

\newcommand{\name}{Marco Söllinger}
\newcommand{\fach}{SEN2}
\newcommand{\topic}{Übung 1}
\newcommand{\uebungangabe}{Uebung1.pdf}

\newcommand{\matnr}{s2410306011}
\newcommand{\uebungsgruppe}{Gruppe 1}
\newcommand{\aufwand}{3}


\pagestyle{fancy}
\normalem 
\fancyhead[R]{Marco Söllinger}  

\begin{document}
\includepdf[pages=1,pagecommand={
			\begin{textblock*}{5cm}(5.5cm, 6.9cm)
				\textbf{\name}
			\end{textblock*}
			\begin{textblock*}{5cm}(5.5cm, 8.4cm)
				\textbf{\matnr}
			\end{textblock*}
			\begin{textblock*}{5cm}(5.5cm, 9.8cm)
				\textbf{\uebungsgruppe}
			\end{textblock*}
			\begin{textblock*}{5cm}(15.3cm, 6.9cm)
				\textbf{\aufwand}
			\end{textblock*}
		}]{Uebung7.pdf}
\includepdf[pages=2-]{Uebung7.pdf}

\section*{Beispiel 1}
Es wird keine Lösungsidee gefordert, deshalb wurde der Code entsprechend kommentiert. \\\\
Das Programm wurde mit einer Makefile kompiliert.\\
Fuer das Testen wurde ein testscript erstellt, welches die verschiedenen Befehle testet.\\\\


\subsection*{1.1 Code}
\lstinputlisting[style=cppstyle, title=\texttt{main.c} ]{../main.c}

\subsection*{1.2 Test}
Es wurden folgende Testfälle implementiert und ausgeführt:

\lstinputlisting[style=cppstyle, title=\texttt{testscript.sh} ]{../testscript.sh}

\begin{lstlisting}[style=cppstyle, title=\texttt{Terminal Output}]
flashfish@fedora ~/D/R/F/B/Uebung07 (main)> ./testscript.sh
clang  -Wall -Wextra -c main.c -o build/main.o
clang -g -o Verzeichnisbaum build/main.o -lm
==> Starte Testlaufe fuer readDir

---------------------------------------------
TEST 0: Invalid Call (kein Argument)(2 Args)
---------------------------------------------
Usage: ./Verzeichnisbaum <directory>

Usage: ./Verzeichnisbaum <directory>

---------------------------------------------
TEST 1: Standardverhalten - aktuelles Verzeichnis
---------------------------------------------
[/home/flashfish/Documents/Repo/FH/BSY3/Uebung07/]
|---[latex/]
| |---[Uebung7.pdf]
| |---[ue9.aux]
| |---[ue9.fdb_latexmk]
| |---[ue9.log]
| |---[ue9.pdf]
| |---[ue9.tex]
| |---[ue9.fls]
| |---[ue9.synctex.gz]
|---[Makefile]
|---[build/]
| |---[main.o]
|---[main.c]
|---[testscript.sh]
|---[Verzeichnisbaum]
|---[test_readDir_env/]

---------------------------------------------
TEST 2: Leeres Verzeichnis
---------------------------------------------
[/home/flashfish/Documents/Repo/FH/BSY3/Uebung07/test_readDir_env/empty/]

---------------------------------------------
TEST 3: Verzeichnis mit Dateien und Unterverzeichnissen
---------------------------------------------
[/home/flashfish/Documents/Repo/FH/BSY3/Uebung07/test_readDir_env/mixed/]
|---[a.txt]
|---[b.bin]
|---[subdir/]

---------------------------------------------
TEST 4: Nicht-existierendes Verzeichnis
---------------------------------------------
chdir (Zielverzeichnis): No such file or directory
Fehler: Pfad 'test_readDir_env/does_not_exist' wurde nicht gefunden.

---------------------------------------------
TEST 5: Pfad zeigt auf Datei statt Verzeichnis
---------------------------------------------
chdir (Zielverzeichnis): Not a directory
Fehler: 'test_readDir_env/file.txt' ist kein Verzeichnis.

---------------------------------------------
TEST 6: Verzeichnis ohne Leserechte (r-Bit fehlt)
---------------------------------------------
[/home/flashfish/Documents/Repo/FH/BSY3/Uebung07/test_readDir_env/noread/]
opendir: Permission denied

---------------------------------------------
TEST 7: Verzeichnis ohne Execute-Rechte (x-Bit fehlt)
---------------------------------------------
chdir (Zielverzeichnis): Permission denied
Fehler: Keine Berechtigung, um 'test_readDir_env/nosearch' zu betreten.

---------------------------------------------
TEST 8: Sehr langer Pfad
---------------------------------------------
Laenge des erzeugten Pfads: 4095 (PATH_MAX ~ 4096)
Aufruf: ./Verzeichnisbaum "test_readDir_env/longpath/subdir_xxxxxxxxxxxxxxxxxxxxxxxxxxxxx/ ... {I removed the rest of the path so that it fits here} ... /subdir_xxxxxxxxxxxxxxxxxxxxxxxxxxxxx/"
getcwd (Zielverzeichnis): Numerical result out of range
Fehler: Laenge des Zielverzeichnispfads ist groesser als PATH_MAX (4096).

---------------------------------------------
TESTS FERTIG
---------------------------------------------
\end{lstlisting}

\end{document}
